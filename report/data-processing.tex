\section{Data Processing}

The data for this project consisted of eleven attributes describing particle
events. Prior to training the models, the data underwent several processing
steps:

\begin{enumerate}
    \item \textbf{Exploration and Visualization:} The data was explored using
    pandas and visualized with matplotlib to understand its distribution,
    identify potential outliers, and assess the presence of class imbalance
    (unequal distribution of gamma and hadron events).
    \item \textbf{Feature Scaling:} If the attributes had significantly
    different scales, feature scaling was applied using techniques like
    standardization or normalization. This ensured all features contributed
    equally to the model's learning process.
    \item \textbf{Imbalanced Class Handling:} If the data exhibited significant
    class imbalance, techniques from imbalanced-learn were employed. These could
    involve oversampling the minority class (gamma events), undersampling the
    majority class (hadron events), or using SMOTE (Synthetic Minority
    Over-sampling Technique) to generate synthetic data points for the minority
    class.
\end{enumerate}
